\apendice{Tecnologias}
\label{ap:tecnologias}

A arquitetura do software foi desenvolvida em três partes: servidor, banco de dados e cliente. Tanto o lado cliente (\textit{frontend}) como o lado servidor(\textit{backend})
utilizam a linguagem Javascript.
Na parte do servidor, ou \textit{server-side}, foram utilizados NodeJS, Express e Mongoose. No lado do cliente foram utilizadas as seguintes tecnologias:
HTML, CSS, Javascript (EcmaScript 6), AngularJS e o Dynagraph. O banco de dados MongoDB foi utilizado como suporte a leitura inicial dos dados.


\section{Servidor}
O \textit{backend} foi utilizado para manipular o banco de dados e transformar as informações na estrutura de dados do Dynagraph para permitir sua análise no \textit{frontend}.
O Node.js é uma plataforma de desenvolvimento para executar código Javascript no \textit{server side} sobre o motor JavaScript do Google Chrome para facilmente construir aplicações de rede rápidas e escaláveis. Uma das maiores vantagens do Node.js é que ele usa um modelo de I/O direcionada a evento não bloqueante que o torna leve e eficiente, ideal para aplicações em tempo real com troca intensa de dados através de dispositivos distribuídos.

O Express.js é um framework JavaScript que facilita a criação de aplicativos web utilizando o Node.js e fornece um conjunto robusto de recursos para aplicativos web e móvel. Com vários métodos utilitários HTTP e \textit{middleware} disponíveis para criar uma API robusta, rápida e fácil. 

Mongoose é uma biblioteca do Nodejs que proporciona uma solução baseada em esquemas para modelar os dados da sua aplicação. Ele possui sistema de conversão de tipos, validação, criação de consultas e \textit{hooks} para lógica de negócios. Mongoose fornece um mapeamento de objetos do MongoDB similar ao ORM (Object Relational Mapping), ou ODM (Object Data Mapping) no caso do Mongoose. Isso significa que o Mongoose traduz os dados do banco de dados para objetos JavaScript para que possam ser utilizados por sua aplicação.

\section{Banco de dados}

MongoDB é um dos mais populares de banco de dados NoSQL. O mesmo é open-source e escrito em C++.
MongoDB é um banco de dados orientado a documentos que armazena dados em documentos JSON com esquema dinâmico. Isso significa que você pode armazenar seus registros sem se preocupar com a estrutura de dados, como o número de campos ou tipos de campos para armazenar valores. Os documentos do MongoDB são semelhantes aos objetos JSON.

\section{Cliente}

HTML

CSS

Javascript, a versão utilizada foi o EcmaScript 6 

AngularJS

Dynagraph

\section{Arquitetura do Softwate}

(TODO - Criar imagem da arquitetura)