\chapter{Fundamentação Teórica}
\label{cap:fundamentacao-teorica}

Referência para DBSCAN
Ester, M., Kriegel, H.-P., Sander, J., and Xu X. (1996), A density-based algorithm for discovering clusters in large spatial data sets with noise. Proc. 2nd Int. Conf. on Knowledge Discovery and Data Mining. Portland, OR, pp.226–231.







Os algoritmos de agrupamento baseados em densidade tentam encontrar clusters com base na densidade de pontos de dados em uma região. A idéia-chave do agrupamento baseado em densidade é que, para cada instância de um cluster, a vizinhança de um determinado raio (Eps) deve conter pelo menos um número mínimo de instâncias (MinPts). Um dos mais conhecidos algoritmos de cluster baseados em densidade é o DBSCAN. DBSCAN separa pontos de dados em três classes (Fig. 2):
• Pontos principais. Estes são pontos que estão no interior de um cluster. Um ponto é um ponto interior se houver pontos suficientes em sua vizinhança.
• Pontos de fronteira. Um ponto de fronteira é um ponto que não é um ponto central, ou seja, não há pontos suficientes em sua vizinhança, mas ele está dentro da vizinhança de um ponto central.
• Pontos de ruído. Um ponto de ruído é qualquer ponto que não é um ponto central ou um ponto de fronteira.

Para encontrar um cluster, o DBSCAN começa com uma instância arbitrária (p) no conjunto de dados (D) e recupera todas as instâncias de D em relação a Eps e MinPts











A Survey on Density Based Clustering Algorithms for Mining Large Spatial Databases

O algoritmo de clustering baseado em densidade é um dos métodos principais para o agrupamento em mineração de dados. Os clusters que são formados com base na densidade são fáceis de entender e não se limitam às formas dos clusters. Este artigo fornece uma pesquisa detalhada dos algoritmos baseados em densidade existente, nomeadamente DBSCAN, VDBSCAN, DVBSCAN, ST-DBSCAN e DBCLASD, com base nos parâmetros essenciais necessários para um bom algoritmo de agrupamento. Analisamos os algoritmos em termos dos parâmetros essenciais para a criação de clusters significativos.

Agrupamentos(introdução)
A mineração de dados é um passo no processo de descoberta de conhecimento em bancos de dados (KDD) consistindo na aplicação de análise de dados e na descoberta de algoritmos que, em limitações de eficiência computacional, produzem uma enumeração particular dos dados [2].
O Gerenciamento de banco de dados espacial (SDBS) [3] são sistemas de banco de dados para o gerenciamento de dados espaciais, ou seja, objetos pontuais ou objetos espacialmente estendidos em um espaço 2D ou 3D ou algum espaço dimensional elevado. A descoberta do conhecimento torna-se cada vez mais importante em termos espaciais, uma vez que quantidades crescentes de dados obtidos a partir de imagens de satélite ou outros equipamentos automáticos são armazenados em bases de dados espaciais.
Spatial Data Mining [4] é o processo de encontrar padrões interessantes e anteriormente desconhecidos, mas potencialmente úteis, de grandes conjuntos de dados espaciais. Extrair padrões de conjuntos de dados espaciais interessantes e úteis são mais difíceis do que extrair o padrão correspondente de dados numéricos e categóricos devido à complexidade de relações espaciais, espaciais e autocorrelação espacial.
Há um crescimento desenfreado de dados espaciais e uma série de necessidades surgem como dados espaciais
técnicas de mineração, modelando propriedades espaciais semânticas ricas, como topologia, modelos de interpretação estatística para padrões espaciais, melhorando a eficiência computacional e modelo, pré-processamento de dados espaciais e muitos outros.
Existem muitas técnicas, como classificação, árvore de decisão, lógica difusa, redes neurais aplicadas para mineração de dados espaciais. A maioria dos trabalhos recentes sobre dados espaciais tem técnicas de agrupamento devido à natureza dos dados.
Clustering, isto é, agrupando os objetos de um banco de dados em subclasses significativas, é um dos principais métodos de mineração de dados [6]. Entre muitos tipos de algoritmos de densidade baseados em cluster. O algoritmo é mais eficiente na detecção de clusters com densidade variada. Houve muita pesquisa sobre algoritmos de agrupamento por décadas, contudo a aplicação a grandes bancos de dados espaciais apresenta os seguintes requisitos:
(i) Número mínimo de parâmetros de entrada. Porque, para grandes bancos de dados espaciais, é muito difícil identificar antecipadamente os parâmetros iniciais, como o número de clusters, forma e densidade.
(ii) Descoberta de clusters com forma arbitrária. Porque a forma dos clusters pode estar em qualquer forma aleatória.
(iii) Uma boa eficiência deve ser alcançada em bancos de dados muito grandes.




Density-Based Algorithms for Discovering Clusters in Large Spatial Databases with Noise (DBSCAN)
DBSCAN [1] é um algoritmo baseado em densidade que descobre clusters com forma arbitrária e com um número mínimo de parâmetros de entrada. Os parâmetros de entrada necessários para este algoritmo são o raio do cluster (Eps) e os pontos mínimos necessários dentro do cluster (Minpts).

2.2. Descrição do Algoritmo
Nesta seção, o algoritmo DBSCAN [7] Clustering espacial baseado em densidade de aplicativos com ruído é projetado para descobrir os clusters de dados espaciais com ruído. As etapas envolvidas neste algoritmo são as seguintes,
…
(i) Selecione um ponto arbitrário p
(ii) Recuperar todos os pontos de densidade-reachable de p w.r.t. Eps e Minpts.
(iii) Se p é um ponto central, um cluster é formado.
(iv) Se p é um ponto de borda, nenhum ponto é densidade acessível de p e DBSCAN visita o próximo ponto do banco de dados.
(v) Continue o processo até que todos os pontos tenham sido processados.

2.3 Impacto do Algoritmo
DBSCAN requer dois parâmetros de entrada (pontos mínimos e raio) e suporta o usuário ao encontrar um valor aproximado para ele usando o gráfico k-dist [7]. Ele descobre grupos de forma arbitrária. Ele é válido para grandes bancos de dados espaciais.
…

2.4 Trabalho futuro
O algoritmo DBSCAN aqui considera [1] apenas objetos de ponto, mas pode ser estendido para outros objetos espaciais, como polígonos. As aplicações do DBSCAN para espaços de recursos de alta dimensão devem ser investigadas e a geração de raio para esses dados de alta dimensão também precisa ser explorada. Também não consegue detectar agrupamentos com densidade variada.



Spatial- Temporal Density Based Clustering (ST-DBSCAN)

6.1. Introdução
…
O algoritmo ST-DBSCAN é construído modificando o algoritmo DBSCAN [7]. Em contraste com o algoritmo de agrupamento baseado em densidade existente, o algoritmo ST-DBSCAN [12] tem a capacidade de descobrir clusters em relação aos valores não espaciais, espaciais e temporais dos objetos. As três modificações feitas no algoritmo DBSCAN são as seguintes,

(i) O algoritmo ST-DBSCAN pode agrupar dados espaciais-temporais de acordo com atributos não espaciais, espaciais e temporais.
(ii) DBSCAN não detecta pontos de ruído quando é de densidade variada, mas isso o algoritmo supera esse problema ao atribuir o fator de densidade a cada cluster.
(iii) Para resolver os conflitos em objetos de borda, ele compara o valor médio de um cluster com o novo valor que vem.

6.2. Descrição do Algoritmo
O algoritmo começa com o primeiro ponto p no banco de dados D.
(i) Este ponto p é processado de acordo com o algoritmo DBSCAN e o próximo ponto é tomado.
(ii) A função RetrieveNeighbors (objeto, Ep1, Ep2) recupera todos os objetos densidade-acessível do objeto selecionado em relação a Eps1, Eps2 e Minpts. Se os pontos devolvidos no Eps-neighborhood são menores do que Minpts, o objeto é atribuído como ruído.
(iii) Os pontos marcados como ruído podem ser alterados posteriormente, e os pontos não são diretamente acessíveis, mas serão densidade-acessível.
…
(iv) Se o ponto selecionado for um objeto central, um novo cluster será construído. Então, todos os vizinhos de densidade direta de este núcleo de objetos também estão incluídos.
(v) Então, o algoritmo coleta de forma iterativa objetos atingidos pela densidade do objeto do núcleo usando a pilha.
(vi) Se o objeto não estiver marcado como ruído ou não estiver em um cluster e a diferença
entre o valor médio do cluster e o novo valor é menor do que DeltaE, ele é colocado no cluster atual.