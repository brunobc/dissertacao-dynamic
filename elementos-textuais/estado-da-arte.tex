% \chapter{Estado da Arte}
\chapter{Conceitos e revisão bibliográfica}
\label{cap:estadodaarte}

Uma visão geral sobre agrupamentos é apresentada na seção \ref{agrupamentos}. Na seção \ref{dbscan} apresentamos o algoritmo DBSCAN em detalhes e suas características. A seção \ref{stdbscan} apresenta o ST-DBSCAN, que é utilizada nesse trabalho para auxlliar os métodos de previsão dinâmica. Na seção  \ref{redes-dinamicas}  discutimos redes dinâmicas, exibimos o modelo DYNAGRAPH e um editor de características. A seção  \ref{trabalhos-relacionados}  define e apresenta os trabalhos relacionados ao problema de agrupamentos dinâmicos.


\section{Agrupamentos}
\label{agrupamentos}

A técnica de agrupamento, também chamada de clustering, é uma das técnicas de mineração de dados mais comuns e é usada para descobrir padrões de distribuição nos dados. O agrupamento é feito com base na similaridade das características e na posição dos objetos. Dessa maneira, o objetivo é que objetos de mesmo grupo sejam muito similares entre si e muito diferentes dos objetos de outros grupos.

Essa técnica é muito utilizada para dados estáticos. No entanto, há pouco trabalho no âmbito espaço-temporal onde os dados estão na forma de campos espaço-temporais contínuos e os agrupamentos são dinâmicos. Além disso, os dados espaço-temporais originados por satélites em órbita terrestre, telefones celulares e outros sensores tendem a ser ruidosos, incompletos e heterogêneos, tornando sua análise especialmente desafiadora \cite{faghmous2013}.

Agrupamentos dinâmicos podem mudar seu tamanho, forma, localização e propriedades estatísticas de um único passo para o próximo. Embora os agrupamentos possam se mover ou mudar de forma, existem vários pontos que não alteram as associações de grupos por um período de tempo. Tendo isso em vista é possível extrair de forma autônoma agrupamentos dinâmicos em dados espaço-temporais contínuos que podem conter valores, ruídos ou características muito variáveis.

Os métodos mais tradicionais são os particionais e os hierárquicos. Alguns algoritmos de agrupamento integram as idéias de vários outros, logo é difícil classificar um algoritmo pertencendo a somente uma categoria de método de agrupamento. Além do que, algumas aplicações podem ter critérios que necessitam a integração de várias técnicas de agrupamento.  Os principais métodos de agrupamento existentes na literatura podem ser categorizados, como mostra as próximas sessões.

\subsection{Métodos baseados em particionamento}
O processo particional consiste em escolher k objetos como sendo os centros dos k grupos, onde k  é um inteiro que representa o número de grupos que se deseja encontrar, que é passado pelo usuário. Além disso, um conjunto de dados D é passado como entrada. 
Os objetos são divididos entre os k grupos de acordo com algum critério de similaridade
estabelecido pelo algoritmo, de modo que cada objeto fique no grupo que tem o menor valor de
distância entre o objeto e o centro dele.
Após a etapa de associação, os centros dos grupos são atualizados, de modo que a função objetivo seja otimizada.

Os algoritmos mais conhecidos dessa abordagem são o k-means e o k-medoids.
Os k-means utilizam a média dos objetos que são do grupo em questão, também conhecido como centro de gravidade do cluster. Os k-medoids representa o objeto que se encontra mais próximo ao centro de gravidade do grupo,
sendo que o elemento mais próximo ao centro se chama medoid.


\subsection{Métodos hierárquicos}
% Dissertação_AntonioCavalcanteAraujoNeto_Final
Os métodos hierárquicos têm como resultado um aninhamento de grupos e o grau de
similaridade entre esses grupos. Os algoritmos hierárquicos podem ser subcategorizados
em aglomerativos e divisivos. Os aglomerativos funcionam de uma maneira botton-up,
assumindo inicialmente que cada elemento do conjunto de dados representa um cluster. A
partir daí, de acordo com algum critério similaridade, os clusters se unem. Já os divisivos
trabalham de maneira top-down, considerando todo o conjunto de dados como um só
cluster, que é dividido de maneira recursiva de acordo com a medida de similaridade
estabelecida.
O algoritmo BIRCH (Balanced Iterative Reducing and Clustering using Hierarchies),
considerado um dos métodos hierárquicos mais utilizados na literatura, faz a integração
de outros métodos, tais como particionamento ou densidade, com organização hierárquica
de clusters. A agrupamento acontece essencialmente em duas fases. Na primeira delas,
uma árvore balanceada é construída através de estruturas chamadas clustering feature
(CF), que sumarizam informações estatísticas sobre os clusters. Essa árvore, chamada de
CF-tree, é utilizada para representar a hierarquia dos clusters. Na segunda fase, um algoritmo
de agrupamento selecionado é aplicado às folhas da CF-tree, removendo clusters
esparsos e agrupando os mais densos em clusters maiores. Uma das maiores vantagens
do algoritmo BIRCH é sua complexidade linear em relação ao número de objetos a ser
agrupado.

\subsection{Métodos baseados em grids}
% Dissertação_AntonioCavalcanteAraujoNeto_Final
Enquanto que os outros métodos de agrupamento discutidos são orientados aos dados,
os métodos baseados em grade são orientados ao espaço. Essencialmente, o espaço
é dividido em células de uma estrutura de grade independente da distribuição dos dados.
Essa estrutura é capaz de representar os objetos do conjunto de entrada através das células
da grade. Todas as operações de agrupamento são realizadas na estrutura de grade
criada. Essa é uma das principais vantagens dessa classe de algoritmos, uma vez que seu
desempenho não depende diretamente do número de entradas do conjunto de dados, mas
sim do número de células em cada dimensão da grade.
O algoritmo STING (STatistical INformation Grid) se baseia na construção de diversas
camadas de grade, onde células de uma camada mais alta são subdivididas para a criação
de células nas camadas mais baixas. A partir dessa organização, informações estatísticas
sobre as células são pré-computadas e armazenadas para a identificação dos clusters.
Os resultados produzidos pelo STING se aproximam do agrupamento produzido pelo
DBSCAN a medida que a granularidade da estrutura de grade se aproxima de 0, podendo
também ser considerado como um método baseado em densidade. O STING também
apresenta como vantagem complexidade linear de tempo em relação ao número de objetos
a serem agrupados.

\subsection{Métodos baseados em densidade}
% Dissertação_AntonioCavalcanteAraujoNeto_Final
Os métodos baseados em densidade assumem que os elementos que pertencem a
um determinado cluster seguem uma mesma distribuição. O objetivo dessas técnicas é
identificar os clusters de acordo com os parâmetros que descrevem essa distribuição. Em
outras palavras, os clusters são modelados como regiões densas do conjunto de dados,
divididos por áreas de regiões esparsas. Dentre as vantagens desses métodos destacase
a capacidade de identificar clusters de formatos arbitrários, enquanto que os métodos
baseados em particionamento, por exemplo, apresentam melhores resultados com clusters
de formato circular.
Dentre os algoritmos baseados em densidade existentes na literatura, o DBSCAN se destaca
como um dos mais utilizados. As características e detalhes do algoritmo DBSCAN
são descritos na Seção 1.

% TODO Spatio-temporal clustering methods

Os agrupamentos baseados em densidade analisam a quantidade de elementos dentro de uma vizinhança de acordo com determinados parâmetros. A idéia-chave é que, para cada instância de um cluster, a vizinhança de um determinado raio deve conter pelo menos um número mínimo de instâncias.
A possibilidade de encontrar agrupamentos de forma eventual e o fato de não precisar da definição do número de agrupamentos \cite{yip2005} como parâmetro inicial são as principais vantagens dos métodos baseados em densidade. Entretanto, alguns algoritmos podem exigir a definição de outros parâmetros, como o caso do algoritmo DBSCAN \cite{density-based-clusters} abordado na próxima seção.

\section{Método DBSCAN}
\label{dbscan}
% TODO DBSCAN KDD96-037
% TODO O metodo dbscan https://www.maxwell.vrac.puc-rio.br/24787/24787_6.PDF

Este algoritmo calcula a densidade de uma região contando quantos pontos existem em uma determinada área seguindo uma determinada métrica, geralmente uma medida de distância, como a euclidiana ou manhattan. O método DBSCAN separa os pontos de dados em três classes:
• Pontos principais. Estes são pontos que estão no interior de um cluster. Um ponto é um ponto interior se houver pontos suficientes em sua vizinhança.
• Pontos de fronteira. Um ponto de fronteira é um ponto que não é um ponto central, ou seja, não há pontos suficientes em sua vizinhança, mas ele está dentro da vizinhança de um ponto central.
• Pontos de ruído. Um ponto de ruído é qualquer ponto que não é um ponto central ou um ponto de fronteira.

Para encontrar um cluster, o DBSCAN começa com uma instância arbitrária (p) no conjunto de dados (D) e recupera todas as instâncias de D em relação a Eps e MinPts
Density-Based Algorithms for Discovering Clusters in Large Spatial Databases with Noise (DBSCAN)
DBSCAN [1] é um algoritmo baseado em densidade que descobre clusters com forma arbitrária e com um número mínimo de parâmetros de entrada. Os parâmetros de entrada necessários para este algoritmo são o raio do cluster (Eps) e os pontos mínimos necessários dentro do cluster (Minpts).

Para agrupar os pontos levando em conta o fator tempo é necessário uma alteração no algoritmo DBScan, e com isso detectar os grupos em relação ao tempo. Logo, o algoritmo determinada para esta implementação foi o ST-DBScan \cite{Birant2007STDBSCANAA}, abordado a seguir.

2.2. Descrição do Algoritmo
Nesta seção, o algoritmo DBSCAN [7] Clustering espacial baseado em densidade de aplicativos com ruído é projetado para descobrir os clusters de dados espaciais com ruído. As etapas envolvidas neste algoritmo são as seguintes,
…
(i) Selecione um ponto arbitrário p
(ii) Recuperar todos os pontos de densidade-reachable de p w.r.t. Eps e Minpts.
(iii) Se p é um ponto central, um cluster é formado.
(iv) Se p é um ponto de borda, nenhum ponto é densidade acessível de p e DBSCAN visita o próximo ponto do banco de dados.
(v) Continue o processo até que todos os pontos tenham sido processados.

2.3 Impacto do Algoritmo
DBSCAN requer dois parâmetros de entrada (pontos mínimos e raio) e suporta o usuário ao encontrar um valor aproximado para ele usando o gráfico k-dist [7]. Ele descobre grupos de forma arbitrária. Ele é válido para grandes bancos de dados espaciais.
…

2.4 Trabalho futuro
O algoritmo DBSCAN aqui considera [1] apenas objetos de ponto, mas pode ser estendido para outros objetos espaciais, como polígonos. As aplicações do DBSCAN para espaços de recursos de alta dimensão devem ser investigadas e a geração de raio para esses dados de alta dimensão também precisa ser explorada. Também não consegue detectar agrupamentos com densidade variada.

% Dissertação_AntonioCavalcanteAraujoNeto_Final
DBSCAN (Density-based Spatial Clustering of Applications with Noise) é um algoritmo
de agrupamento baseado em densidade vastamente utilizado pela comunidade científica.
Seu objetivo principal consiste em encontrar concentrações de elementos que estão espacialmente
próximos. Em outras palavras, o algoritmo busca por pontos que possuem mais que um
limiar de vizinhos dentro de um certo raio. Caso um elemento p satisfaça essa propriedade, os
vizinhos de p pertencerão ao mesmo cluster que p e o mesmo processo é aplicado a todos os
seus vizinhos. Além do conjunto de dados a ser agrupado, o DBSCAN recebe também dois
parâmetros de entrada: minPoints e eps. O primeiro deles se refere à quantidade mínima de
pontos em um certo raio de vizinhança para a formação de um cluster. Já o segundo parâmetro
se refere ao raio no qual a verificação de vizinhança é realizada. A função de distância utilizada
para determinar a vizinhança de um certo ponto é definida de acordo com o tipo de dado
a ser agrupado e deve obedecer às restrições de uma função de distância, tais como simetria
e a desigualdade triangular, além de assumir que a distância entre dois elementos x e y só é
igual a 0 se x = y. Dentre suas vantagens, o algoritmo DBSCAN se destaca por ser capaz de
encontrar clusters com formatos arbitrários, além de ser capaz de lidar com ruídos nos dados,
característica que não está presente na maioria dos algoritmos de agrupamento, como mostrado
na Figura 2.1.
As definições básicas utilizadas no DBSCAN são apresentadas a seguir:
• |A|: cardinalidade do conjunto A.
• Neps(p): é o conjunto de pontos q que estão a uma distância menor que eps do ponto p.
Também é chamado de conjunto dos vizinhos de p.
• Diretamente alcançável por densidade (DDR): um ponto p é DDR a partir de um ponto q
se p E Neps(q) e |Neps(q)| >= minPoints.

Figura 2.1: Clusters de formatos arbitrários encontrados pelo algoritmo DBSCAN
Fonte: Data Mining - The Hypertextbook
• Alcançável por densidade (DR): um ponto p é DR a partir de um ponto q, se existe uma
sequência de pontos {p1,..., pn} onde p1 = p e pn = q, tal que pi+1 é DDR a partir de pi
.
• Conectado por densidade (DC): Um ponto p está conectado por densidade a um ponto q
se existe um ponto o tal que p e q são DR a partir de o.
• Core point: um ponto p é classificado como core point se |Neps(o)| >= minPoints.
• Border point: um ponto p é classificado como border point se |Neps(p)| < minPoints e p
é DDR a partir de um core point.
• Noise: Um ponto p é classificado como noise se |Neps(q)| < minPoints e p não é DDR a
partir de nenhum core point.
No contexto do algoritmo DBSCAN, um cluster C é definido como um subconjunto
não vazio dos dados que satisfaz as seguintes propriedades:
• Maximalidade: Para quaisquer dois pontos p e q, se p E C e q é alcançável por densidade
(DR) a partir de p, então q E C.
• Conectividade: Para quaisquer dois pontos p,q E C, p é q são conectados por densidade
(DC).
O Algoritmo 1 mostra o pseudocódigo do DBSCAN. Para cada elemento p ainda
não visitado o conjunto dos seus vizinhos Neps(p) é encontrado, como podemos ver entre as linhas
2 e 5. Caso a cardinalidade desse conjunto de vizinhos seja maior que o valor de minPoints
(Linha 6 do Algoritmo), um novo cluster C é criado e p e seus vizinhos serão atribuídos a C.
Ainda, os pontos não visitados de C serão expandidos em um processo similar. A agrupamento
acaba quando todos os elementos do conjunto foram visitados.
O Algoritmo 2 implementa a função de expansão de um cluster C. A expansão
de um cluster a partir de um ponto p encontra todos os elementos que são conectados por
densidade (DC) a p. Essa função recebe como entrada p, seu conjunto de vizinhos NeighborPts,
o identificador C do cluster a ser expandido, e os parâmetros eps e minPoints. Para cada ponto
p' do conjunto de vizinhos de p, caso esse ponto ainda não tenha sido visitado, sua vizinhança
é recuperada e adicionada ao conjunto NeighborPts de vizinhos que serão verificados (linha
8). Após sua verificação, caso p' não pertença a nenhum cluster, ele é adicionado ao cluster C
(linhas 11 a 13). O processo finaliza quando o conjunto NeighborPts está vazio.

Alg 1 e alg 2

Como podemos ver nos Algoritmos 1 e 2, a complexidade do DBSCAN depende
diretamente do custo computacional para recuperação da vizinhança de um ponto (linhas 7 e
6 dos Algoritmos 1 e 2, respectivamente). Em uma solução ingênua essa operação poderia ser
executada em tempo linear, onde uma simples busca exaustiva em todo o conjunto de dados
retornaria apenas os elementos a uma certa distância do ponto de consulta. Tal solução faria
com que a complexidade do algoritmo DBSCAN fosse O(n2). Por outro lado, com o auxílio
de estruturas de índices, como k-d-Trees ou R-Trees, a complexidade do DBSCAN pode ser
significantemente reduzida. Recentemente foi provado em (GAN; TAO, 2015) que para dimensões
maiores que 2 o algoritmo DBSCAN executa em uma complexidade Omega(n4/3). No entanto,
consideraremos nesse trabalho conjuntos de dados de apenas duas dimensões.

\section{Método ST-DBSCAN}
\label{stdbscan}
% Spatial- Temporal Density Based Clustering (ST-DBSCAN)

6.1. Introdução

O algoritmo ST-DBSCAN é construído modificando o algoritmo DBSCAN [7]. Em contraste com o algoritmo de agrupamento baseado em densidade existente, o algoritmo ST-DBSCAN [12] tem a capacidade de descobrir clusters em relação aos valores não espaciais, espaciais e temporais dos objetos. As três modificações feitas no algoritmo DBSCAN são as seguintes,

(i) O algoritmo ST-DBSCAN pode agrupar dados espaciais-temporais de acordo com atributos não espaciais, espaciais e temporais.
(ii) DBSCAN não detecta pontos de ruído quando é de densidade variada, mas isso o algoritmo supera esse problema ao atribuir o fator de densidade a cada cluster.
(iii) Para resolver os conflitos em objetos de borda, ele compara o valor médio de um cluster com o novo valor que vem.

6.2. Descrição do Algoritmo
O algoritmo começa com o primeiro ponto p no banco de dados D.
(i) Este ponto p é processado de acordo com o algoritmo DBSCAN e o próximo ponto é tomado.
(ii) A função RetrieveNeighbors (objeto, Ep1, Ep2) recupera todos os objetos densidade-acessível do objeto selecionado em relação a Eps1, Eps2 e Minpts. Se os pontos devolvidos no Eps-neighborhood são menores do que Minpts, o objeto é atribuído como ruído.
(iii) Os pontos marcados como ruído podem ser alterados posteriormente, e os pontos não são diretamente acessíveis, mas serão densidade-acessível.
…
(iv) Se o ponto selecionado for um objeto central, um novo cluster será construído. Então, todos os vizinhos de densidade direta de este núcleo de objetos também estão incluídos.
(v) Então, o algoritmo coleta de forma iterativa objetos atingidos pela densidade do objeto do núcleo usando a pilha.
(vi) Se o objeto não estiver marcado como ruído ou não estiver em um cluster e a diferença
entre o valor médio do cluster e o novo valor é menor do que DeltaE, ele é colocado no cluster atual.

\section{Redes dinâmicas}
 \label{redes-dinamicas} 
\subsection{O modelo Dynagraph}
\subsection{Editor de características}
\section{Trabalhos Relacionados}
 \label{trabalhos-relacionados} 
Como há uma carência de estudos relacionando os assuntos abordados: agrupamento,
previsão em dados dinâmicos espaço-temporais, grafos dinâmicos e sistemas web
de forma integrada, foi necessário dividir o problema de agrupamentos e previsões dinâmicos em três etapas:
\begin{itemize}
\item Estrutura de dados em grafos dinâmicos
\item Modelos de previsão espaço-temporais
\item Algoritmos de agrupamentos dinâmicos
\end{itemize}

A pesquisa aborda estrutura de dados em grafos dinâmicos usando passos já descritos na literatura,
principalmente o modelo Dynagraph \cite{dynagraph}, que é baseado na primeira proposta
em \cite{dynagraph2012}, onde o Dynagraph usa sequências temporais para vértices, arestas,
características modificáveis dos vértices e arestas e o relacionamento entre suas características.
Com isso, é formado um grafo com as informações necessárias para qualquer instante no tempo.
O Dynagraph é capaz de visualizar o comportamento do grafo ao longo de um período de tempo,
e editá-lo.

A ideia central de \cite{kim} é modelar uma rede dinâmica como digrafos orientados ao
tempo (\textit{time-ordered graph}), que é gerada através da ligação de instantes temporais com arestas
direcionadas que unem cada nó ao seu sucessor no tempo. Com isso, transformar uma rede dinâmica
em um grafo maior, mas facilmente analisável. Isto permite não só a utilização dos algoritmos 
desenvolvidos para grafos estáticos, mas também para melhor definir métricas para grafos dinâmicos.
Segundo \cite{kim} um sistema de grafos dinâmicos é um objeto de representação visual
que pode descrever melhor o comportamento dinâmico de objetos relacionados a eventos dinâmicos e
introduzir novas formas de enxergar ou descrever a evolução de eventos dinâmicos na natureza.

\cite{kostakos} considera a estrutura de grafos temporais como grafos
estáticos, no entanto avança sobre as métricas introduzindo conceitos como disponibilidade
temporal, proximidade temporal e geodésica, e estuda os seus grafos sobre redes reais.

Segundo \cite{density-based-clusters}, o algoritmo DBScan(\textit{Density-Based Spatial Clustering
of Applications With Noise}) calcula a densidade de uma região contando quantos pontos existem
em uma determinada área seguindo uma determinada métrica. Ele permite a redução de pontos não
pertencentes a nenhum padrão, assim como possibilita a formação de grupos de diferentes formas.
Seu objetivo principal é dividir os pontos em grupos através da densidade de cada região.

\cite{lahiri2007} apresentam um algoritmo de predição em redes temporais, e que usa a ideia de que certas
interações sinalizam a ocorrência de outros em algum momento no futuro. Através de análises estatísticas
o algoritmo mede o atraso entre as interações, e com isso pode-se prever quando certas interações vão ocorrer
com base em observações passadas e atuais. Propõe-se a utilização de subgrafos frequentes e discute
como identificar subgrafos que são persistidos em redes temporais.
\cite{lahiri2008} em seguida propõe um novo problema de mineração de dados para redes dinâmicas:
detecção de todos os padrões de interação que ocorrem em intervalos de tempo regulares.

% \cite{alfredo} propôs um algoritmo baseado no método IGN (Identificador
% de Grupos Naturais) de \cite{simposioNeg}, onde este apresenta bons resultados tanto para distâncias 
% euclidianas quanto para outras distâncias; é sensível à presença de \textit{outliers} em situações muito específicas;
% e o número de grupos naturais e sua composição é obtida automaticamente no processo.
% A pesquisa seguirá este método para verificar o processo de construção de agrupamentos dinâmicos.










