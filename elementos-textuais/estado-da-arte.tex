% \chapter{Estado da Arte}
\chapter{Conceitos e revisão bibliográfica}
\label{cap:estadodaarte}

\section{Agrupamentos}
Os dados espaço-temporais estão rapidamente se tornando onipresentes graças a sensores e armazenamento acessíveis. Estes dados ricos em informações têm o potencial de revolucionar diversos campos, como as ciências sociais, terrestres e médicas, onde há necessidade de extrair e compreender fenômenos espaço-temporais complexos e suas dinâmicas. Além disso, os dados em tais domínios científicos tendem a ser grandes e não marcados. Isso destaca a importância de métodos não supervisionados no monitoramento de dinâmicas espaço-temporais com pouca ou nenhuma supervisão humana.

Clustering é uma das técnicas de mineração de dados não supervisionadas mais comuns. Tem um enorme sucesso, especialmente para dados estáticos. No entanto, há pouco trabalho na configuração espaço-temporal onde os dados estão na forma de campos espaço-temporais contínuos e os clusters são dinâmicos. Além disso, os dados espaço-temporais originados por satélites em órbita terrestre, telefones celulares e outros sensores tendem a ser ruidosos, incompletos e heterogêneos, tornando sua análise especialmente desafiadora.
\cite{faghmous2013}
[Faghmous and Kumar, 2013]
James H Faghmous and Vipin
Kumar. Spatio-temporal data mining for climate data: Advances, challenges, and opportunities. In W. Chu, editor, Data Mining and Knowledge Discovery for Big Data: Methodologies, Challenges, and Opportunities, pages 83–116. Springer, 2013.

TODO melhorar texto
A técnica de agrupamento espacial, também chamada como clustering, é muito útil na
área de mineração de dados e é usada para descobrir padrões de distribuição nos dados. O
agrupamento é feito com base na similaridade das características e na posição dos objetos.
Dessa maneira, o objetivo é que objetos de mesmo agrupamento sejam muito similares entre
si e muito diferentes dos objetos de outros agrupamentos (JIN; MIAO, 2010) (KIM et al, 2014).

Neste trabalho, propomos um novo paradigma de agrupamento espaço-temporal para identificar clusters em um campo espaço-temporal contínuo onde os clusters são dinâmicos e podem mudar seu tamanho, forma, localização e propriedades estatísticas de um único passo para o próximo. O nosso paradigma decorre da observação de que, em inúmeras configurações dinâmicas, embora os agrupamentos possam se mover ou mudar de forma, existem vários pontos que não altere as associações de grupos para um período de tempo significativo.
Esta observação nos permite extrair de forma autônoma clusters dinâmicos em dados espaço-temporais contínuos que podem conter valores, ruídos ou características muito variáveis.

//OK
Os agrupamentos baseados em densidade analisam a quantidade de elementos dentro
de uma vizinhança de acordo com determinados parâmetros. 
 A possibilidade de encontrar clusters de forma arbitrárias e o fato de não precisar da definição do número de clustes \cite{yip2005} como parâmetro inicial são as principais vantagens dos métodos baseados em densidade. Entretanto, alguns algoritmos podem exigir a definição de outros parâmetros, como o caso do algoritmo DBSCAN \cite{density-based-clusters} abordado na próxima seção.

\section{Agrupamento por densidade}
TODO Spatio-temporal clustering methods

\subsection{Método DBSCAN}
TODO DBSCAN KDD96-037
TODO O metodo dbscan https://www.maxwell.vrac.puc-rio.br/24787/24787_6.PDF
\subsection{Método ST-DBSCAN}
\section{Redes dinâmicas}
\subsection{O modelo Dynagraph}
\subsection{Editor de características}
\section{Trabalhos Relacionados}

Como há uma carência de estudos relacionando os assuntos abordados: agrupamento,
previsão em dados dinâmicos espaço-temporais, grafos dinâmicos e sistemas web
de forma integrada, foi necessário dividir o problema de agrupamentos e previsões dinâmicos em três etapas:
\begin{itemize}
\item Estrutura de dados em grafos dinâmicos
\item Modelos de previsão espaço-temporais
\item Algoritmos de agrupamentos dinâmicos
\end{itemize}

A pesquisa aborda estrutura de dados em grafos dinâmicos usando passos já descritos na literatura,
principalmente o modelo Dynagraph \cite{dynagraph}, que é baseado na primeira proposta
em \cite{dynagraph2012}, onde o Dynagraph usa sequências temporais para vértices, arestas,
características modificáveis dos vértices e arestas e o relacionamento entre suas características.
Com isso, é formado um grafo com as informações necessárias para qualquer instante no tempo.
O Dynagraph é capaz de visualizar o comportamento do grafo ao longo de um período de tempo,
e editá-lo.

A ideia central de \cite{kim} é modelar uma rede dinâmica como digrafos orientados ao
tempo (\textit{time-ordered graph}), que é gerada através da ligação de instantes temporais com arestas
direcionadas que unem cada nó ao seu sucessor no tempo. Com isso, transformar uma rede dinâmica
em um grafo maior, mas facilmente analisável. Isto permite não só a utilização dos algoritmos 
desenvolvidos para grafos estáticos, mas também para melhor definir métricas para grafos dinâmicos.
Segundo \cite{kim} um sistema de grafos dinâmicos é um objeto de representação visual
que pode descrever melhor o comportamento dinâmico de objetos relacionados a eventos dinâmicos e
introduzir novas formas de enxergar ou descrever a evolução de eventos dinâmicos na natureza.

\cite{kostakos} considera a estrutura de grafos temporais como grafos
estáticos, no entanto avança sobre as métricas introduzindo conceitos como disponibilidade
temporal, proximidade temporal e geodésica, e estuda os seus grafos sobre redes reais.

Segundo \cite{density-based-clusters}, o algoritmo DBScan(\textit{Density-Based Spatial Clustering
of Applications With Noise}) calcula a densidade de uma região contando quantos pontos existem
em uma determinada área seguindo uma determinada métrica. Ele permite a redução de pontos não
pertencentes a nenhum padrão, assim como possibilita a formação de grupos de diferentes formas.
Seu objetivo principal é dividir os pontos em grupos através da densidade de cada região.

\cite{lahiri2007} apresentam um algoritmo de predição em redes temporais, e que usa a ideia de que certas
interações sinalizam a ocorrência de outros em algum momento no futuro. Através de análises estatísticas
o algoritmo mede o atraso entre as interações, e com isso pode-se prever quando certas interações vão ocorrer
com base em observações passadas e atuais. Propõe-se a utilização de subgrafos frequentes e discute
como identificar subgrafos que são persistidos em redes temporais.
\cite{lahiri2008} em seguida propõe um novo problema de mineração de dados para redes dinâmicas:
detecção de todos os padrões de interação que ocorrem em intervalos de tempo regulares.

% \cite{alfredo} propôs um algoritmo baseado no método IGN (Identificador
% de Grupos Naturais) de \cite{simposioNeg}, onde este apresenta bons resultados tanto para distâncias 
% euclidianas quanto para outras distâncias; é sensível à presença de \textit{outliers} em situações muito específicas;
% e o número de grupos naturais e sua composição é obtida automaticamente no processo.
% A pesquisa seguirá este método para verificar o processo de construção de agrupamentos dinâmicos.











