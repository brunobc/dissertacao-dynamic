% \chapter{Estado da Arte}
\chapter{Referencial Teórico}
\label{cap:estadodaarte}

Como há uma carência de estudos relacionando os assuntos abordados: agrupamento,
previsão em dados dinâmicos espaço-temporais, grafos dinâmicos e sistemas web
de forma integrada, foi necessário dividir o problema de agrupamentos e previsões dinâmicos em três etapas:
\begin{itemize}
\item Estrutura de dados em grafos dinâmicos
\item Modelos de previsão espaço-temporais
\item Algoritmos de agrupamentos dinâmicos
\end{itemize}

A pesquisa aborda estrutura de dados em grafos dinâmicos usando passos já descritos na literatura,
principalmente o modelo Dynagraph \cite{dynagraph}, que é baseado na primeira proposta
em \cite{dynagraph2012}, onde o Dynagraph usa sequências temporais para vértices, arestas,
características modificáveis dos vértices e arestas e o relacionamento entre suas características.
Com isso, é formado um grafo com as informações necessárias para qualquer instante no tempo.
O Dynagraph é capaz de visualizar o comportamento do grafo ao longo de um período de tempo,
e editá-lo.

A ideia central de \cite{kim} é modelar uma rede dinâmica como digrafos orientados ao
tempo (\textit{time-ordered graph}), que é gerada através da ligação de instantes temporais com arestas
direcionadas que unem cada nó ao seu sucessor no tempo. Com isso, transformar uma rede dinâmica
em um grafo maior, mas facilmente analisável. Isto permite não só a utilização dos algoritmos 
desenvolvidos para grafos estáticos, mas também para melhor definir métricas para grafos dinâmicos.
Segundo \cite{kim} um sistema de grafos dinâmicos é um objeto de representação visual
que pode descrever melhor o comportamento dinâmico de objetos relacionados a eventos dinâmicos e
introduzir novas formas de enxergar ou descrever a evolução de eventos dinâmicos na natureza.

\cite{kostakos} considera a estrutura de grafos temporais como grafos
estáticos, no entanto avança sobre as métricas introduzindo conceitos como disponibilidade
temporal, proximidade temporal e geodésica, e estuda os seus grafos sobre redes reais.

Segundo \cite{density-based-clusters}, o algoritmo DBScan(\textit{Density-Based Spatial Clustering
of Applications With Noise}) calcula a densidade de uma região contando quantos pontos existem
em uma determinada área seguindo uma determinada métrica. Ele permite a redução de pontos não
pertencentes a nenhum padrão, assim como possibilita a formação de grupos de diferentes formas.
Seu objetivo principal é dividir os pontos em grupos através da densidade de cada região.

\cite{lahiri2007} apresentam um algoritmo de predição em redes temporais, e que usa a ideia de que certas
interações sinalizam a ocorrência de outros em algum momento no futuro. Através de análises estatísticas
o algoritmo mede o atraso entre as interações, e com isso pode-se prever quando certas interações vão ocorrer
com base em observações passadas e atuais. Propõe-se a utilização de subgrafos frequentes e discute
como identificar subgrafos que são persistidos em redes temporais.
\cite{lahiri2008} em seguida propõe um novo problema de mineração de dados para redes dinâmicas:
detecção de todos os padrões de interação que ocorrem em intervalos de tempo regulares.

\cite{alfredo} propôs um algoritmo baseado no método IGN (Identificador
de Grupos Naturais) de \cite{simposioNeg}, onde este apresenta bons resultados tanto para distâncias 
euclidianas quanto para outras distâncias; é sensível à presença de \textit{outliers} em situações muito específicas;
e o número de grupos naturais e sua composição é obtida automaticamente no processo.
A pesquisa seguirá este método para verificar o processo de construção de agrupamentos dinâmicos.











