% \chapter{Estado da Arte}
\chapter{Conceitos e revisão bibliográfica}
\label{cap:estadodaarte}

Uma visão geral sobre agrupamentos é apresentada na seção \ref{agrupamentos}. Na seção \ref{dbscan} apresentamos o algoritmo DBSCAN em detalhes e suas características. A seção \ref{stdbscan} apresenta o ST-DBSCAN, que é utilizada nesse trabalho para auxiliar os métodos de previsão dinâmica. Na seção  \ref{redes-dinamicas}  discutimos redes dinâmicas, exibimos o modelo DYNAGRAPH e um editor de características. A seção  \ref{trabalhos-relacionados}  define e apresenta os trabalhos relacionados ao problema de agrupamentos dinâmicos.


\section{Agrupamentos}
\label{agrupamentos}

A técnica de agrupamento, também chamada de \textit{clustering}, é uma das técnicas de mineração de dados mais comuns e é usada para descobrir padrões de distribuição nos dados. O agrupamento é feito com base na similaridade das características e na posição dos objetos. Dessa maneira, o objetivo é que os objetos do mesmo grupo sejam muito similares entre si e muito diferentes dos objetos de outros grupos.

Essa técnica é muito utilizada para dados estáticos. No entanto, há pouco trabalho no âmbito espaço-temporal onde os dados estão na forma de campos espaço-temporais contínuos e os agrupamentos são dinâmicos. Além disso, os dados espaço-temporais originados por satélites em órbita terrestre, telefones celulares e outros sensores tendem a ser ruidosos, incompletos e heterogêneos, tornando sua análise especialmente desafiadora \cite{faghmous2013}.

Agrupamentos dinâmicos podem mudar seu tamanho, forma, localização e propriedades estatísticas de um único passo para o próximo. Embora os agrupamentos possam se mover ou mudar de forma, existem vários pontos que não alteram as associações de grupos por um período de tempo. Tendo isso em vista é possível extrair de forma autônoma agrupamentos dinâmicos em dados espaço-temporais contínuos que podem conter valores, ruídos ou características muito variáveis.

Os métodos mais tradicionais são os particionais e os hierárquicos. Alguns algoritmos de agrupamento integram as idéias de vários outros, logo é difícil classificar um algoritmo pertencendo a somente uma categoria de método de agrupamento. Além do que, algumas aplicações podem ter critérios que necessitam a integração de várias técnicas de agrupamento.  Os principais métodos de agrupamento existentes na literatura podem ser categorizados, como mostram as próximas sessões.

\subsection{Métodos baseados em particionamento}
A ideia principal desta classe de algoritmos de agrupamentos é criar ${K}$ grupos dos dados, onde ${K}$ é inserido pelo usuário.
Esse método consiste em escolher ${K}$ objetos como sendo os centros dos ${K}$ grupos.
Os objetos são divididos entre os ${K}$ grupos de acordo com algum critério de similaridade
estabelecido pelo algoritmo, de modo que cada objeto fique no grupo que tem o menor valor de
distância entre o objeto e o centro do grupo a que ele pertence.
Os algoritmos de particionamento são muito populares devido à sua facilidade de implementação e baixo custo computacional; no entanto, eles têm essas desvantagens: (1) eles são sensíveis à presença de ruído e outliers, (2) eles podem descobrir apenas grupos com formas convexas e (3) o número de grupos precisa ser especificado.

\subsubsection{Agrupamentos K-Médias}
O algoritmo de agrupamento k-médias é uma das técnicas mais populares dessa abordagem e foi introduzido em \cite{Macqueen67}. Ele utiliza a média dos objetos que são do grupo em questão, também conhecido como centro de gravidade do grupo.  A idéia principal neste algoritmo é usar a média dos objetos para atribuí-los a grupos e também usá-los para  representar estes.
K-médias é um algoritmo que garante a convergência para um ótimo local, mas não necessariamente um ótimo global. À medida que ${K}$ aumenta, o custo de encontrar a solução ótima diminui, e atinge seu mínimo quando ${K}$ é igual ao número de objetos \cite{Wu2008}. Especificamente, o procedimento é mostrado abaixo:
 \begin{enumerate}
	\item Inserir os objetos para serem agrupados e também o número ${K}$ de grupos.
	\item Escolher aleatoriamente os ${K}$ objetos como centro dos grupos originais.
	\item Atribuir cada objeto ao grupo com a média mais próxima.
	\item Calcular a nova média de cada grupo.
	\item Repetir a partir do passo 3.
	\item Parar quando o critério de convergência estiver satisfeito. Outro critério, mais frequentemente usado, é a minimização do erro quadrático, dado por: 
	\begin{equation}
	E = \sum_{i=1}^{k}\sum_{o\in C_{i}} |o - \mu_{i}|^{2}
	\end{equation}
	onde ${o}$ é o ponto no espaço representando um dado objeto, ${\mu_{i}}$ é o representante do grupo ${C_{i}}$,
	 e ${K}$ o número de grupos.
\end{enumerate}

TODO --- Represente a evolução deste método em figuras.
\subsubsection{Agrupamentos K-Medoids}
O algoritmo de K-Medoids foi introduzido primeiramente em \cite{kaufmann1990} e não é tão sensível aos outliers quanto os k-médias. Nesse algoritmo, cada grupo é representado pelo objeto mais próximo ao centro, conhecido como medoid.
O processo geral para o algoritmo é o seguinte:
 \begin{enumerate}
 	\item Escolher aleatoriamente ${k}$ objetos como os medoids iniciais.
 	\item Atribuir cada um dos objetos restantes ao grupo que possui o medoid mais próximo.
 	\item Em um grupo, selecionar aleatoriamente um objeto que não seja medoid (${nonmedoid}$), que será referenciado como ${O_{nonmedoid}}$.
 	\item Calcular o custo de substituir o medoid com ${O_{nonmedoid}}$. Este custo é a diferença no erro quadrático se o medoid atual for substituído por ${O_{nonmedoid}}$. Se for negativo, faça ${O_{nonmedoid}}$ o medoid do grupo. O erro quadrático é novamente somado ao erro de todos os objetos:
 	\begin{equation}
 	E = \sum_{i=1}^{k}\sum_{o\in C_{i}} |o - O_{medoid(i)}|^{2}
 	\end{equation}
 	onde ${O_{medoid(i)}}$ é o medoid do grupo ${C_{i}}$.
 	\item Repetir a partir do passo 2 até que não haja mudanças. 
 \end{enumerate}
TODO --- Represente a evolução deste método em figuras.

\subsection{Métodos hierárquicos}
Nesta classe de algoritmos os objetos são colocados em uma hierarquia que é percorrida de uma forma bottom-up ou top-down para criar os grupos. A vantagem deste tipo de agrupamento é que ele não requer nenhum conhecimento sobre o número de grupos, e sua desvantagem é sua complexidade computacional \cite{Lin2004}. Muitas vezes, uma estrutura em árvore, um dendrograma, é usada para representar os níveis hierárquicos aninhados.

Os aglomerativos funcionam de uma maneira botton-up,
assumindo inicialmente que cada elemento do conjunto de dados representa um grupo. Em seguida, os grupos são fundidos em grupos maiores até a criação de um grupo único ou qualquer outro critério de parada.

A abordagem divisisa trabalham de maneira top-down, considerando todo o conjunto de dados como um só
grupo, que é dividido de maneira recursiva de acordo com a medida de similaridade
estabelecida.

\subsubsection{O algoritmo BIRCH}

Em \cite{Zhang1996}, BIRCH significa Balanced Iterative Reducing and Clustering using Hierarchies sendo um algoritmo usado para agrupamentos em bases de dados muito grandes. É considerado um dos métodos hierárquicos mais utilizados na literatura. As vantagens do BIRCH são as seguintes:

\begin{itemize}
	\item BIRCH trabalha de maneira local. Isso é obtido usando medições que indicam a proximidade natural dos pontos, de modo que cada decisão de agrupamento possa ser feita sem verificar todos os pontos de dados ou grupos existentes.
	\item Leva em conta a estrutura de dados espacial. Ele trata os pontos em uma região densa como um único grupo, enquanto os pontos em uma região esparsa são caracterizados como outliers e podem ser removidos opcionalmente.
	\item O algoritmo faz uso total da memória disponível enquanto minimiza os custos de Entrada/saída de dados.
\end{itemize}

Os principais conceitos do BIRCH, que funcionam de maneira incremental, são o Agrupamento por característica(AC) e AC-árvore. Onde AC consiste em todas as informações que precisam ser mantidas sobre um grupo.
Já AC-árvore é utilizada para representar a hierarquia de grupos.
O agrupamento acontece essencialmente em duas fases. Na primeira delas, o algoritmo lê o conjunto de dados e constrói uma AC-árvore inicial. Essa árvore é utilizada para representar a hierarquia dos grupos. Na segunda fase, um algoritmo de agrupamento selecionado é aplicado às folhas da AC-árvore, removendo grupos esparsos e agrupando os mais densos em grupos maiores.

\subsubsection{O algoritmo CURE}

Em \cite{Guha1998}, um novo algoritmo hierárquico é proposto para detectar grupos que não são necessariamente convexos. Os grupos do CURE são representados por um número fixo de pontos bem espalhados que são encolhidos em direção ao centro do grupo por uma determinada fração. O CURE difere do algoritmo BIRCH de duas maneiras:
\begin{itemize}
\item O CURE começa desenhando uma amostra aleatória em vez de pré-agrupar todos os pontos de dados, como no caso do BIRCH.
\item CURE primeiro particiona a amostra aleatória e, em seguida, em cada partição, os dados são parcialmente agrupados. Em seguida, os outliers são eliminados e os dados pré-agrupados em cada partição são agrupados para gerar os grupos finais.
\end{itemize}
Os resultados experimentais no artigo mostram que o tempo de execução do CURE é sempre menor que o do BIRCH. Mais importante, os resultados mostram que, à medida que o tamanho do banco de dados aumenta, o tempo de execução do BIRCH aumenta rapidamente enquanto o tempo de execução do CURE aumenta muito pouco. O motivo é que o BIRCH varre todo o banco de dados e usa todos os pontos para o pré-agrupamento, enquanto o CURE usa apenas uma amostra aleatória.

\subsection{Métodos baseados em estrutura de grade}
Enquanto que os outros métodos de agrupamento discutidos são orientados aos dados, os métodos baseados em grade são orientados ao espaço. Essencialmente, o espaço é dividido em células retangulares, representadas por uma estrutura de grade hierárquica.

\cite{Wang1997} propuseram um método de agrupamento baseado em grade, STING (STatistical INformation Grid - Informação estatística baseada em grade), para agrupar bancos de dados espaciais e facilitar consultas orientadas à região. Esse algoritmo se baseia na construção de diversas camadas de grade, onde células de uma camada mais alta são subdivididas para a criação de células nas camadas mais baixas, como mostra a figura \ref{fig:sting}.

\begin{figure}[!h]
	\centering	
	\Caption{\label{fig:sting} Algoritmo STING: subdivisão de células e construção de árvores}	
	\UECEfig{}{
		\includegraphics[width=8cm]{figuras/sting.png}
	}{
		\Fonte{\cite{Berkhin2002}}
	}
\end{figure}

O desempenho de STING depende da granularidade do nível mais baixo da estrutura de grade e o resultado dos
grupos são limitados, pois só crescem na horizontal ou vertical e sofrem para buscar grupos de formatos complexos.

Os resultados produzidos pelo STING se aproximam do agrupamento produzido pelo
DBSCAN a medida que a granularidade da estrutura de grade se aproxima de 0, podendo
também ser considerado como um método baseado em densidade, que será descrito a seguir.
Uma das vantagens do STING é a complexidade linear de tempo em relação ao número de objetos
a serem agrupados.

\subsection{Métodos baseados em densidade}
Nesta classe de algoritmos, a idéia principal é manter os grupos em crescimento, desde que sua densidade esteja acima de um certo limite. A vantagem dos algoritmos baseados em densidade, em comparação com os algoritmos de particionamento baseados em distância, é que eles podem detectar grupos de forma arbitrária. Isso também fornece uma proteção natural contra outliers. Por outro lado, os algoritmos baseados em distância detectam apenas aglomerados de forma convexa.

Os agrupamentos baseados em densidade analisam a quantidade de elementos dentro de uma vizinhança de acordo com determinados parâmetros. A idéia-chave é que, para cada instância de um grupo, a vizinhança de um determinado raio deve conter pelo menos um número mínimo de instâncias.

A possibilidade de encontrar agrupamentos de forma eventual e o fato de não precisar da definição do número de agrupamentos \cite{yip2005} como parâmetro inicial são as principais vantagens dos métodos baseados em densidade. Entretanto, alguns algoritmos podem exigir a definição de outros parâmetros, como o caso do algoritmo DBSCAN \cite{ESTER1996} abordado na próxima seção.

\section{Método DBSCAN}
\label{dbscan}
% TODO DBSCAN KDD96-037
% TODO O metodo dbscan https://www.maxwell.vrac.puc-rio.br/24787/24787_6.PDF

Este método calcula a densidade de uma região contando quantos pontos existem em uma determinada área seguindo uma determinada métrica, geralmente uma medida de distância, como a euclidiana ou manhattan. 
É um método efetivo para identificar grupos de formato arbitrário e de diferentes tamanhos, separar os ruídos dos dados, requer apenas um parâmetro de entrada, ajuda o usuário na determinação de um valor apropriado para ele e ajuda a detectar grupos e seus arranjos dentro do espaço de dados, sem qualquer informação preliminar sobre os grupos.
\cite{ESTER1996}  escrevem que a noção de agrupamentos e o algoritmo DBSCAN se aplicam para espaços euclidianos de duas e três dimensões, como para qualquer espaço característico de alta dimensão. O método DBSCAN é aplicável a qualquer base de dados contendo dados de um espaço métrico, isto é, bases de dados com uma função de distância para pares de objetos. Finalmente, o DBSCAN é eficiente mesmo para grandes bancos de dados espaciais.
Para entender o método é necessário conhecer alguns conceitos básicos do algoritmo: 

\begin{enumerate}
	\item Vizinhança: Determinada pela função de distância, que pode ser distância euclidiana ou distância manhattan, para dois pontos ${p}$ e ${q}$, dado por ${dist(p,q)}$.
	\item Eps: raio ao redor de um ponto. 
	\item Eps-vizinhança: a vizinhança de um ponto ${p}$ com raio Eps, dado por ${N_{Eps}(p)}$, é definido por ${N_{Eps}(p) = \big\{ q \in D | dist(p, q)  \leqslant Eps\big\} }$. Na figura \ref{fig:epsViz} abaixo os círculos representam respectivamente a vizinhança Eps dos pontos ${q}$ e ${p}$.
	\item Ponto Central: Se o Eps-vizinhança de um ponto ${p}$ contém ao menos um número mínimo, MinPts, de pontos, então o ponto ${p}$ é chamado de ponto central.
	Por exemplo, na figura \ref{fig:epsViz}, se for adotado MinPts = 4, ${p}$ é um ponto central e os demais não são pontos centrais.
	\item Ponto de Borda: Se a Eps-vizinhança de um ponto ${p}$ contém algum ponto central e menos que MinPts então o ponto ${p}$ é chamado de ponto de borda. Na figura  \ref{fig:epsViz}, ${q}$, ${r}$ e ${s}$ são pontos de borda.
	\item Diretamente alcançável: um ponto ${p}$ é alcançável se ele está no Eps-vizinhança de ${q}$ e este é um ponto central.
	\item Maximalidade: um ponto ${p}$ é alcançável se existe uma cadeia de pontos que os liga, que respeita Eps e MinPts.
	\item Conectividade: Um ponto ${p}$ está conectado a um outro ${q}$ se existe um objeto ${o}$ tal que tanto ${p}$ quanto ${q}$ são alcançáveis a partir de ${o}$, com respeito a Eps e MinPts. Logo é uma relação simétrica.
	\item Cluster: Seja D uma base de dados, um grupo C é um subconjunto não vazio de D, respeitando Eps e MinPts, onde:
		\subitem ${\forall}$ ${p}$, ${q}$ se ${q}$ ${\in}$ C e ${p}$ é alcançável a partir de ${q}$.
		\subitem ${\forall}$ ${p}$, ${q}$ ${\in}$ C ${p}$ está conectado por densidade a ${q}$.
\end{enumerate}

\begin{figure}[!h]
	\centering
	\Caption{\label{fig:epsViz} Eps-vizinhança de  ${q}$ e Eps-vizinhança de ${p}$}	
	\UECEfig{}{
		\includegraphics[width=8cm]{figuras/epsViz.png}
	}{
		\Fonte{Elaborado pelo autor}
	}
\end{figure}

O procedimento para encontrar um agrupamento é baseado no fato de que um grupo é
inequivocamente determinado por qualquer de seus centros \cite{ESTER1998}. 
O DBSCAN pode utilizar R*-tree para obter um melhor desempenho, mesmo assim  a complexidade média do tempo de execução do DBSCAN será ${O (N logN)}$, onde ${N}$ é o número de objetos da base de dados \cite{Sheikholeslami1998}.

\subsection{Algoritmo DBSCAN}

As etapas envolvidas neste algoritmo são as seguintes:

\begin{enumerate}
	\item Selecionar um ponto arbitrário ${p}$;
	\item Recuperar todos os pontos que são alcançáveis por densidade a partir do ponto ${p}$;
	\item Se ${p}$ é um ponto central, um grupo é formado;
	\item Se ${p}$ é um ponto de borda, nenhum ponto é alcançável por densidade a partir de ${p}$ e DBSCAN visita o próximo ponto do banco de dados;
	\item Continuar o processo até que todos os pontos tenham sido processados.
\end{enumerate}

% ver https://www.maxwell.vrac.puc-rio.br/24787/24787_6.PDF

\begin{algorithm}[h!]
	\SetSpacedAlgorithm
	\caption{\label{alg:algoritmo_dbscan}Algoritmo DBScan}
	\Entrada{D, Eps1, MinPts, Delta, Epsilon}
	\Saida{C}
	\Inicio{
			\Para{i = 0 até n}{
				\Se{${o_i}$ nao foi visitado}{
					marcar ${o_i}$ como visitado\;
					Vizinhos = BuscarVizinhos(${o_i}$)\;
					\Se {|Vizinhos| < MinPts}{
						MarcarRuido(${o_i}$)\;
					}
				    \Senao{
				    	C = próximo Cluster\;
				    	adicionar ${o_i}$ ao cluster C\;
				    	expandirCluster(${o_i}$, C, Vizinhos)\;
				    }
				}
			}
	}
\end{algorithm}

\begin{algorithm}[h!]
	\SetSpacedAlgorithm
	\caption{\label{alg:algoritmo_dbscan_exp}Algoritmo DBScan - Expandir Cluster}
	\Entrada{Vizinhos, C, MinPts}
	\Inicio{
			\Para{i = 0 até número de Vizinhos}{
				${p}$ = ${Vizinhos_{(i)}}$\;
				\Se {${p}$ ainda não foi visitado}{
					marcar ${p}$ como visitado\;
					Vizinhos2 = BuscarVizinhos(p)\;
					
					\Se {Vizinhos2 ${\geqslant}$ MinPts}{
						Vizinhos = Vizinhos ${\cup}$ Vizinhos2\;
					}
				}
				\Se{${p}$ não está em nenhum cluster}{
					adicionar ${p}$ ao cluster C\;
				}
				
			}
	}
\end{algorithm}


\section{Método ST-DBSCAN}
\label{stdbscan}

% Comentar um bloco de código: \iffalse \fi
Para agrupar os pontos levando em conta o fator tempo é necessário uma alteração no algoritmo DBSCAN \cite{ESTER1998}, e com isso detectar os grupos em relação ao tempo. Esse algoritmo usa apenas um parâmetro de distância Eps para medir a similaridade de dados espaciais com uma dimensão. A fim de suportar dados espaciais bidimensionais, o ST-DBSCAN \cite{Birant2007STDBSCANAA} usa duas métricas de distância, Eps1 e Eps2, para definir a similaridade por uma conjunção de dois testes de densidade.
O algoritmo ST-DBSCAN é construído modificando o algoritmo DBSCAN. Em contraste com o algoritmo de agrupamento baseado em densidade existente, o algoritmo ST-DBSCAN tem a capacidade de descobrir agrupamentos de acordo com valores não espaciais, espaciais e temporais dos objetos. As três modificações feitas no algoritmo DBSCAN são as seguintes:
\begin{itemize}
\item Permitir o algoritmo ST-DBSCAN descobrir grupos em dados espaciais-temporais.
\item Introdução do fator de densidade atribuído a cada grupo, que é o seu grau de densidade,
para encontrar objetos de ruído quando existem grupos de diferentes densidades.
\item A terceira modificação fornece uma comparação do valor médio de um grupo com o novo valor resultante. Necessária para resolver os conflitos em pontos de borda.
\end{itemize}

O algoritmo DBSCAN não é satisfatório quando existem agrupamentos de diferentes densidades. Para superar esse problema,  o ST-DBSCAN atribui a cada grupo um fator de densidade, que é o grau da densidade do grupo.
A função DensityDistance é dada por:

${DensityDistance = \frac{DensityDistanceMax}{DensityDistanceMin}}$
\linebreak
onde DensityDistanceMax de um objeto ${p}$ denota a distância máxima entre o objeto ${p}$ e seus vizinhos dentro do raio Eps. Da mesma forma, DensityDistanceMin de um objeto ${p}$ denota a distância mínima entre o objeto ${p}$ e seus vizinhos dentro do raio Eps.

${DensityDistanceMax(p) = max\big\{ dist(p, q) | q \in D \wedge dist(p, q)  \leqslant Eps\big\} }$

${DensityDistanceMin(p) = min\big\{ dist(p, q) | q \in D \wedge dist(p, q)  \leqslant Eps\big\} }$
\linebreak

O fator de densidade de um grupo ${C}$ é dado por:

${DensityFactor(C) = 1\big/\left [   \frac{\sum_{p\in C}DensityDistance(p)}{|C|} \right ]
}$
\linebreak

Com esse fator é possível evitar problemas de agrupamentos com grande variação de densidade, onde os grupos são pequenos e muito densos ou grandes e pouco densos.

\subsection{Algoritmo ST-DBSCAN}
Uma grande diferença entre o DBSCAN e o ST-DBSCAN é a utilização de um segundo parâmetro EPS, onde é usado para medir a similaridade de valores não espaciais. Nesta pesquisa usa-se o tempo como EPS de valor não espacial. Para dois pontos serem considerados vizinhos eles devem respeitar os limites de espaço e tempo parametrizados por EPS1 e EPS2.

\begin{enumerate}
	\item O algoritmo começa com o primeiro ponto ${p}$ no banco de dados D.
	\item Este ponto ${p}$ é processado de acordo com o algoritmo DBSCAN e o próximo ponto é tomado.
	\item A função BuscarVizinhos(p, Ep1, Ep2) recupera todos os pontos que tem uma distância menor que Eps1 e Eps2 para o ponto ${p}$. Se os pontos recuperados não podem ser alcançados o ponto é atribuído como ruído, onde o ponto selecionado não tem vizinhos suficientes para ser armazenado em grupo.
	\item Os pontos marcados como ruído podem ser alterados posteriormente, se não forem diretamente alcançáveis pela densidade, mas eles são alcançáveis por densidade de algum outro ponto do banco de dados. Isso acontece para pontos de borda de um grupo.
	\item Se o ponto selecionado tiver vizinhos suficientes dentro das distâncias Eps1 e Eps2 - se for um ponto central - então um novo grupo será construído.
	\item Todos os vizinhos diretamente atingíveis por densidade desse ponto central também são incluídos.
	\item O algoritmo reúne iterativamente pontos acessíveis por densidade a partir desse ponto central usando uma pilha.
	\item Se o objeto não estiver marcado como ruído ou não estiver em um grupo e a diferença entre o valor médio do grupo e o novo valor é menor do que o valor limite para incluir em um grupo, ${\Delta E}$, ele é colocado no grupo atual.
	\item Se dois grupos ${C_1}$ e ${C_2}$ estão muito próximos um do outro, um ponto ${p}$ pode pertencer a ambos ${C_1}$ e ${C_2}$. Então o ponto ${p}$ é atribuído ao grupo que foi descoberto primeiro.
	\item Depois de processar o ponto selecionado, o algoritmo seleciona o próximo ponto em D e o algoritmo continua iterativamente até que todos os pontos tenham sido processados.
\end{enumerate}

\pagebreak
\begin{algorithm}[h!]
	\SetSpacedAlgorithm
	\caption{\label{alg:algoritmo_stdbscan}Algoritmo ST-DBScan}
	\Entrada{D, Eps1, Eps2, MinPts, ${\Delta E}$}
	\Saida{C = (${C_1}$, ${C_2}$, ... ${C_k}$) Conjunto de clusters}
	\Inicio{
		\Para{i = 1 até n}{
			\Se {${o_i}$ não está no cluster}{
				Vizinhos = BuscarVizinhos(${o_i}$, Eps1, Eps2)\;
				\Se {|Vizinhos| < MinPts}{
					marcar ${p}$ como ruído\;
				}
			   \Senao{
			   		\Para{j = 1 até |Vizinhos|}{
			   			RotularPontos(${X_j}$)\;
			   			AdicionarPontosAoCluster(${X_j}$, ${C_i}$)\;
			   		}
		   		\Enqto{existir pontos na vizinhança de ${o_i}$}{
		   			pontoAtual = vizinho(${o_i}$)\;
		   			Vizinhos2 = BuscarVizinhos(pontoAtual, Eps1, Eps2)\;
		   			\Se{|Vizinhos2| >= MinPts}{
		   				\Para{ todo ${o_y}$}{
		   					\Se{(${\neg}$ehRuido(${o_y}$) ${\vee}$ ${\neg}$estaNumGrupo(${o_y}$)) ${\wedge}$ DensidadeMediaDoGrupo($C_i$) +${o_y}$ > ${\Delta E}$ }{
		   						AdicionarPontosAoCluster(${o_y}$, ${C_i}$)\;
		   					}
		   				}
		   			}
		   		}
		       }
			}
		}
	}
\end{algorithm}

TODO -- criar subsections 
\cite{lahiri2007} apresentam um algoritmo de predição em redes temporais, e que usa a ideia de que certas
interações sinalizam a ocorrência de outros em algum momento no futuro. Através de análises estatísticas
o algoritmo mede o atraso entre as interações, e com isso pode-se prever quando certas interações vão ocorrer
com base em observações passadas e atuais. Propõe-se a utilização de subgrafos frequentes e discute
como identificar subgrafos que são persistidos em redes temporais.
\cite{lahiri2008} em seguida propõe um novo problema de mineração de dados para redes dinâmicas:
detecção de todos os padrões de interação que ocorrem em intervalos de tempo regulares.

\cite{alfredo2009} propôs um algoritmo baseado no método IGN (Identificador de Grupos Naturais) de \cite{simposioNeg2003}, onde este apresenta bons resultados tanto para distâncias euclidianas quanto para outras distâncias; é sensível à presença de \textit{outliers} em situações muito específicas; e o número de grupos naturais e sua composição é obtida automaticamente no processo.



\subsection{Agrupamento de dados baseado em predição}
%Temporal Data Mining [Mitsa]
A tarefa de predição visa descobrir o valor futuro de um determinado atributo de dados. É uma área com variedade de aplicações em diversas áreas como meteorologia e detecção de doenças. Por exemplo, um médico gostaria de prever a reação de seus pacientes a um novo medicamento para diabetes, particularmente a duração dos episódios de hipoglicemia.
Na previsão de eventos, é desejável prever a ocorrência de um evento ou o número de ocorrências de um evento ou a duração de um evento, dada a existência de certas condições. Por exemplo, um médico acaba de colocar um de seus pacientes epilépticos em uma nova droga que é muito eficaz, mas após o início da terapia pode causar um grave ataque de enxaqueca. O médico gostaria de prever a duração desse ataque, considerando o conhecimento sobre a idade do paciente e o número e duração dos ataques epilépticos no último ano. Nos problemas de previsão de eventos que lidam com a previsão da duração de um evento pode ser modelada usando uma variável contínua e pode-se usar a regressão linear para sua previsão, onde a regressão linear é um dos métodos de regressão mais amplamente disponíveis \cite{Mitsa:2010}.
Na previsão de séries temporais, os dados são dados históricos obtidos em intervalos de tempo regulares. Informações sobre padrões passados podem ser usadas para prever padrões futuros. No exemplo da enxaqueca, os dados poderiam ser a duração dos ataques de enxaqueca de outros pacientes sobre a droga, juntamente com informações sobre suas idades, condições médicas preexistentes e gravidade da epilepsia.

\section{Redes dinâmicas}
 \label{redes-dinamicas}
 
 TODO --- Introduzir o processo de previsão de agrupamentos espaço temporais e sugira dois métodos básicos de previsão: Usar o livro Mitsa (2010) pagina 105 (clsutering of Time Series Data Steams, Time Series Clustering Using Global Characteristics. Mostre como cada um funciona com um exemplo. Vá nos artigos que apresentam estes métodos---

 
 Para redes dinâmicas, o foco é tipicamente em uma classe de redes e questões referentes às estrutura dessa classe de rede, como a estrutura evoluiu e como isso afeta sistemas dinâmicos na rede. As redes temporais normalmente são mais orientadas a dados - onde se investiga um conjunto de dados, suas estruturas e como, por exemplo, surtos epidêmicos se comportariam sobre ele. Em seguida, é questionado como essas observações se generalizam comparando os resultados para diferentes conjuntos de dados \cite{holme:colloquium}.
 
\subsection{O modelo Dynagraph}

O Dynagraph \cite{dynagraph} é um modelo computacional que permite, a partir de uma estrutura de dados simples, modelo de um Grafo ou Rede Dinâmica, representar com o mínimo custo de armazenamento a evolução de uma instância de observação e estudos. Ele acompanha a evolução dos conjuntos de um grafo: de nós e ligações (arcos e elos), como inserção/retirada ao longo do tempo, e mudança de suas características (posição, cor, forma, tamanho, e outras), por meio de um editor de características, apresentado na próxima subseção.

\subsubsection{Estrutura de dados}

(TODO - anexar imagem da estrutura JSON usada pelo Dynagraph)

\subsection{Editor de características}
O Editor de Características é uma extensão do software DYNAGRAPH, que permite alterar os atributos visuais dos vértices e aresta de um grafo dinâmico.
Com ele podemos criar novos tipos de dados e com isso diferenciar, por exemplo, focos de casos de dengue e tipos de dengue.
Essa extensão permite criar com os seguintes atributos: Rótulo, Opacidade, Escala, Espessura da Borda, Cor da Borda e Cor de Preenchimento. Outra opção é utilizar uma imagem.
 
A figura \ref{fig:edCaMenu} exibe a localização do Editor de Características na aplicação em um menu lateral com as seguinte opções: Vértice, Aresta e um painel, este exibe a data atual e número de vértices e arestas em análise.
\begin{figure}[!h]
	\centering	
	\caption{\label{fig:edCaMenu} Editor de Características: menu lateral}
	\UECEfig{}{
		\includegraphics[width=15cm]{figuras/editorCaract/edCaractMenu.png}
	}{
		\Fonte{Elaborado pelo autor}
	}
\end{figure}
\FloatBarrier

Para criar um novo vértice deve-se selecionar o submenu relacionado a vértices como mostra a figura \ref{fig:edCaCriacao1}.
\begin{figure}[!h]
	\centering	
	\Caption{\label{fig:edCaCriacao1} Editor de Características: criação de um vértice}	
	\UECEfig{}{
		\includegraphics[width=15cm]{figuras/editorCaract/1edCaractCreateVertice.png}
	}{
		\Fonte{Elaborado pelo autor}
	}
\end{figure}
\FloatBarrier

A figura \ref{fig:edCaCriacao2} exibe os tipos de dados permitidos na criação ou edição de um vértice.
\begin{figure}[h]
	\centering	
	\Caption{\label{fig:edCaCriacao2} Editor de Características: criação de um vértice - tipos de dados}	
	\UECEfig{}{
		\includegraphics[width=15cm]{figuras/editorCaract/2edCaractCreateVertice.png}
	}{
		\Fonte{Elaborado pelo autor}
	}
\end{figure}
\FloatBarrier

Após criar um novo tipo de vértice podemos usá-lo editando um dado ponto no mapa, como mostram as figuras \ref{fig:edCaChangeV1} e \ref{fig:edCaChangeV2}.
A figura \ref{fig:edCaShowV} exibe o resultado esperado.
\begin{figure}[h]
	\centering	
	\Caption{\label{fig:edCaChangeV1} Editor de Características: edição de um vértice - parte 1}	
	\UECEfig{}{
		\includegraphics[width=15cm]{figuras/editorCaract/3edCaractChangeVertice.png}
	}{
		\Fonte{Elaborado pelo autor}
	}
\end{figure}
\FloatBarrier

\begin{figure}[h]
	\centering	
	\Caption{\label{fig:edCaChangeV2} Editor de Características: edição de um vértice - parte 2}	
	\UECEfig{}{
		\includegraphics[width=15cm]{figuras/editorCaract/4edCaractChangeVertice.png}
	}{
		\Fonte{Elaborado pelo autor}
	}
\end{figure}
\FloatBarrier

\begin{figure}[h]
	\centering	
	\Caption{\label{fig:edCaShowV} Editor de Características: edição de um vértice - parte 3}	
	\UECEfig{}{
		\includegraphics[width=15cm]{figuras/editorCaract/5edCaractShowVertice.png}
	}{
		\Fonte{Elaborado pelo autor}
	}
\end{figure}
\FloatBarrier

Para editar um tipo de vértice basta selecioná-lo a partir da lista de tipos de vértices e então aplicar as mudanças, como mostra a figura \ref{fig:edCaEditV1} e o resultado na figura \ref{fig:edCaEditV2}.
\begin{figure}[h]
	\centering	
	\Caption{\label{fig:edCaEditV1} Editor de Características: edição de um tipo de vértice - parte 1}	
	\UECEfig{}{
		\includegraphics[width=15cm]{figuras/editorCaract/6edCaractEditVertice.png}
	}{
		\Fonte{Elaborado pelo autor}
	}
\end{figure}
\FloatBarrier
\begin{figure}[h]
	\centering	
	\Caption{\label{fig:edCaEditV2} Editor de Características: edição de um tipo de vértice - parte 2}	
	\UECEfig{}{
		\includegraphics[width=15cm]{figuras/editorCaract/7edCaractEditedVertice.png}
	}{
		\Fonte{Elaborado pelo autor}
	}
\end{figure}
\FloatBarrier

Outra opção de edição de um vértice é dado na figura \ref{fig:edCaractEditVerticeOption2}. Nesta opção podemos usar uma imagem pronta.
\begin{figure}[h]
	\centering	
	\Caption{\label{fig:edCaractEditVerticeOption2} Editor de Características: Tipo de vértice}	
	\UECEfig{}{
		\includegraphics[width=15cm]{figuras/editorCaract/edCaractEditVerticeOption2.png}
	}{
		\Fonte{Elaborado pelo autor}
	}
\end{figure}
\FloatBarrier

O Editor de Características permite também a criação e edição de arestas (figura \ref{fig:edCaractAresta}), mas essa funcionalidade não foi necessária nesta pesquisa.
\begin{figure}[h]
	\centering	
	\Caption{\label{fig:edCaractAresta} Editor de Características: Aresta}	
	\UECEfig{}{
		\includegraphics[width=15cm]{figuras/editorCaract/edCaractAresta.png}
	}{
		\Fonte{Elaborado pelo autor}
	}
\end{figure}
\FloatBarrier

\subsection{Visualização dos grupos formados}
Foram usados dois algoritmos para permitir a visualização dos grupos formados em um dado tempo.
O primeiro é uma biblioteca da API do Google chamada MarkerClusterer \cite{markerCluster} que cria e gerencia grupos de acordo com o nível de zoom para grandes quantidades de pontos.
Essa biblioteca é combinada com a API Javascript do Google Maps para agrupar os pontos por proximidade em grupos e simplificar a exibição dos pontos no mapa.
De acordo com o nível de zoom os grupos são formados com cores e tamanhos diferentes, como mostra as figuras \ref{fig:apiGoogleMaps1} e \ref{fig:apiGoogleMaps2}.
\begin{figure}[h]
	\centering	
	\Caption{\label{fig:apiGoogleMaps1} Grupos - API Google Maps parte 1}	
	\UECEfig{}{
		\includegraphics[width=13cm]{figuras/ClusterAPIGoogle1.png}
	}{
		\Fonte{Elaborado pelo autor}
	}
\end{figure}
\FloatBarrier

\begin{figure}[h]
	\centering	
	\Caption{\label{fig:apiGoogleMaps2} Grupos - API Google Maps parte 2}	
	\UECEfig{}{
		\includegraphics[width=13cm]{figuras/ClusterAPIGoogle2.png}
	}{
		\Fonte{Elaborado pelo autor}
	}
\end{figure}
\FloatBarrier

O segundo algoritmo é conhecido como Convex Hull \cite{ConvexHull}. Também conhecido como fecho convexo onde, dado um conjunto de pontos em um espaço,
o problema consiste em encontrar o menor número de pontos que gerem um polígono convexo no qual abranja todos os outros pontos.
As figuras \ref{fig:algConvexHull1} e \ref{fig:algConvexHull2} são exemplos do convex hull, onde cada polígono representa o grupo de casos de dengue naquela região delimitada. Dos vários pontos, somente os pontos mais externos e que formam o menor polígono que englobam todos os outros pontos.
%
\begin{figure}[h]
	\centering	
	\Caption{\label{fig:algConvexHull1} Exemplo 1 de Convex Hull}	
	\UECEfig{}{
		\includegraphics[width=14cm]{figuras/algConvexHull1.png}
	}{
		\Fonte{Elaborado pelo autor}
	}
\end{figure}
\FloatBarrier
\begin{figure}[h]
	\centering	
	\Caption{\label{fig:algConvexHull2} Exemplo 2 de Convex Hull}	
	\UECEfig{}{
		\includegraphics[width=14cm]{figuras/algConvexHull2.png}
	}{
		\Fonte{Elaborado pelo autor}
	}
\end{figure}
\FloatBarrier

\pagebreak
\section{Trabalhos Relacionados}
 \label{trabalhos-relacionados} 
Como há uma carência de estudos relacionando os assuntos abordados: agrupamento,
previsão em dados dinâmicos espaço-temporais, grafos dinâmicos e sistemas web
de forma integrada, foi necessário dividir o problema de agrupamentos e previsões dinâmicos em três etapas:
\begin{itemize}
\item Estrutura de dados em grafos dinâmicos
\item Modelos de previsão espaço-temporais
\item Algoritmos de agrupamentos dinâmicos
\end{itemize}

A pesquisa aborda estrutura de dados em grafos dinâmicos usando passos já descritos na literatura,
principalmente o modelo Dynagraph \cite{dynagraph}, que é baseado na primeira proposta
em \cite{dynagraph2012}, onde o Dynagraph usa sequências temporais para vértices, arestas,
características modificáveis dos vértices e arestas e o relacionamento entre suas características.
Ele permite formar um grafo com as informações necessárias para qualquer instante no tempo.
O Dynagraph é capaz de visualizar o comportamento do grafo ao longo de um período de tempo,
e editá-lo.

A ideia central de \cite{kim} é modelar uma rede dinâmica como digrafos orientados ao
tempo (\textit{time-ordered graph}), que é gerada através da ligação de instantes temporais com arestas
direcionadas que unem cada nó ao seu sucessor no tempo. Com isso, transformar uma rede dinâmica
em um grafo maior, mas facilmente analisável. Isto permite não só a utilização dos algoritmos 
desenvolvidos para grafos estáticos, mas também para melhor definir métricas para grafos dinâmicos.
Segundo \cite{kim} um sistema de grafos dinâmicos é um objeto de representação visual
que pode descrever melhor o comportamento dinâmico de objetos relacionados a eventos dinâmicos e
introduzir novas formas de enxergar ou descrever a evolução de eventos dinâmicos na natureza.

\cite{kostakos} considera a estrutura de grafos temporais como grafos
estáticos, no entanto avança sobre as métricas introduzindo conceitos como disponibilidade
temporal, proximidade temporal e geodésica, e estuda os seus grafos sobre redes reais.












