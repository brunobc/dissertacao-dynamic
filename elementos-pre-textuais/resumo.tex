Este trabalho apresenta uma abordagem para solucionar o problema da previsão de agrupamentos
dinâmicos espaço-temporal. Foram implementados 2 métodos para visualização dos grupos formados.
O primeiro é uma biblioteca que cria e gerencia grupos de acordo com o nível de zoom.
O segundo, chamado algoritmo Convex Hull, consiste em gerar o menor polígono que englobe um determinado
conjunto de pontos. O algoritmo ST-DBSCAN foi implementado como base para alcançar o método proposto.
Por último, foram desenvolvidos 2 aprimoramentos no software DYNAGRAPH para
possibilitar a avaliação dos métodos. Um deles foi a integração com um Editor de
Características, que permite alterar os atributos visuais dos vértices e arestas de um
grafo dinâmico. Outro novo recurso permitiu a visulalização da formação de novos grupos dinâmicos.

% Separe as palavras-chave por ponto
\palavraschave{Agrupamento Dinâmicos, Grafos dinâmicos.}