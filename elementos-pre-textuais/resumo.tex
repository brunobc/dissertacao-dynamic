Arboviroses são doenças causadas pelos chamados arbovírus, que incluem o vírus da dengue, Zika vírus, febre chikungunya e febre amarela. Essas doenças estão cada vez mais voltando a atenção da OMS-TDR e das autoridades de saúde brasileiras para um esforço maior na prevenção e combate a endemias. Esta pesquisa fez importantes esforços para desenvolver, projetar e implementar uma estrutura computacional baseada na web, para ajudar a rastrear e gerenciar os recursos e pessoas no processo de prevenção e combate à arbovírus. Além disso, apresenta uma abordagem para solucionar o problema da previsão de agrupamentos dinâmicos espaço-temporal e combater as arboviroses com o esforço coordenado de uma estrutura de Sistemas de Apoio à Decisão para rastrear simultaneamente o mosquito e os casos em humanos, para prevenir e combater os territórios afetados.
 Foram implementados 2 métodos para visualização dos grupos formados.
O primeiro é uma biblioteca que cria e gerencia grupos de acordo com o nível de zoom.
O segundo, chamado algoritmo Convex Hull, consiste em gerar o menor polígono que englobe um determinado
conjunto de pontos. O algoritmo ST-DBSCAN foi implementado como base para alcançar o método proposto.
Por último, foram desenvolvidos 2 aprimoramentos no software Dynagraph para
possibilitar a avaliação dos métodos. Um deles foi a integração com um Editor de
Características, que permite alterar os atributos visuais dos vértices e arestas de um
grafo dinâmico. Outro novo recurso permitiu a visulalização da formação de novos grupos dinâmicos.
TODO - predição
% Ao longo de quinze anos, acompanhamos o desenvolvimento e o surto da dengue nas cidades de Fortaleza e do Rio de Janeiro. Um grupo de pesquisadores brasileiros de diferentes áreas: medicina / epidemiologia, matemática, estatística, pesquisa de operações e ciência da computação fez importantes esforços para desenvolver, projetar e implementar uma estrutura computacional baseada na web, para ajudar a rastrear e gerenciar os recursos e pessoas o processo de prevenção e combate à dengue. A evolução cíclica da doença de 2003 a 2017 e, mais recentemente, a incorporação de novas doenças mais perigosas, Zyka e Chikungunya, confirmam o nome “Arbovírus” para doenças provocadas pelo mosquito Aedes Aegypty. Isso está cada vez mais voltando a atenção da OMS-TDR e das autoridades de saúde brasileiras para um esforço mais prevenção e combate. 

%Os modelos MP consideram problemas bem conhecidos e novas soluções para Clustering restrito, Roteamento de Arco, Agendamento e Cluster Dinâmico com componente preditivo espaço-temporal. A evolução das ferramentas utilizadas reais indica que o sucesso dessa coordenação, com e sem a presença do nosso DSS, passa pela adoção da tecnologia DSS baseada em programação matemática.

% Separe as palavras-chave por ponto
\palavraschave{Agrupamento Dinâmicos, Grafos dinâmicos.}